% tipo
\documentclass{article}

% formato
\usepackage[letterpaper, margin = 1.5cm]{geometry}
\usepackage{float}

% matematicas
\usepackage{amsmath}
\usepackage[spanish]{babel}
\usepackage[utf8]{inputenc}
\usepackage{amssymb}
\usepackage{ebproof}
% encabezado

\title{
    Lenguajes de Programación 2020-1\\
    Facultad de Ciencias UNAM\\
    Ejercicio Semanal 8
}

\author{
    Sandra del Mar Soto Corderi\\
    Edgar Quiroz Castañeda
}

\date{
    10 de octubre del 2019
}

\begin{document}
    \maketitle

    \begin{enumerate}
        \item {
		   Encontrar el tipo mas general para las siguientes expresiones de 
		   Cálculo Lambda utilizando el algoritmo de inferencia de tipos W.
        	
        	\begin{enumerate}
        		\item {
				 $\texttt{and} = \lambda x\lambda y.xy\texttt{true} = 
				 \lambda x\lambda y.xy (\lambda x\lambda y.x) 
				 =_{\alpha} \lambda x\lambda y.xy (\lambda a\lambda b.a)$
        		 
        		 Como Var($\texttt{and}$) = $\{x$: X, $y$: Y, $a$: A, $b$: B$\}$
        		 \begin{align}        		 
        		 &\varnothing | \Gamma \models x : X &\mathrm{(Var)}\\
        		 &\varnothing | \Gamma \models y : Y &\mathrm{(Var)}\\
        		 &= 0 &\mathrm{(Definicion \ Integral)}
        		 \end{align}
        		}
        		\item {
					$\texttt{or} = \lambda x \lambda y.x \texttt{true} y$
					 
					Primero
					\begin{figure}[H]
						\centering
						\begin{prooftree}

							\hypo{x:X \in \Gamma_1 = \{x:X\}}
							\infer 1 [Var] {
								\varnothing | \Gamma_1 \vdash x:X
							}

							\infer 0 [Ax] {
								\varnothing | \varnothing
								\vdash \texttt{true}:Bool
							}

							\hypo{S = \varnothing}

							\infer [no rule] 1 {Z_1 \texttt{ fresh}}

							\infer [no rule] 1 {
								(\varnothing \cup \{X\})
								\cap
								(\varnothing \cup \varnothing)
								= \varnothing
							}

							\infer 3 [App] {
								R_1 = \{X = \texttt{Bool} \mapsto Z_1\} 
								| \Gamma_1
								\vdash x \texttt{true} : Z_1
							}
						\end{prooftree}
						\label{A}
					\end{figure}

					Luego
					\begin{figure}[H]
						\centering
						\begin{prooftree}
							\hypo{Anterior}
							\infer 1 [App] {
								R_1 | \Gamma_1 \vdash x \texttt{true} : Z_1
							}

							\hypo{y:Y \in \Gamma_2 = \{y:y\}}
							\infer 1 [Var] {
								\varnothing | \Gamma_2 \vdash y:Y
							}

							\hypo{S = \{X = \texttt{Bool} \mapsto Z_1\}}

							\infer [no rule] 1 {Z_2 \texttt{ fresh}}

							\infer [no rule] 1 {
								(\{X, Z_1\} \cup \{X\})
								\cap
								(\varnothing \cup \varnothing)
								= \varnothing
							}

							\infer 3 [App] {
								R_2 = \{Z_1 = Y \mapsto Z_2\} \cup R_1
								| \Gamma_3 = \Gamma_1 \cup \Gamma_2 \vdash x \texttt{true} y : Z_2
							}

							\infer 1 [App] {
								R_2
								| \Gamma_3 \vdash x \texttt{true} y : Z_2
							}

							\infer 1 [Lam] {
								R_2
								| \Gamma_4 = \Gamma_3 \backslash \{y : Y\} 
								\vdash \lambda y . x \texttt{true} y : Y \mapsto Z_2
							}

							\infer 1 [Lam] {
								R_2
								| \varnothing = \Gamma_4 \backslash \{x : X\} 
								\vdash \lambda x. \lambda y . 
								x \texttt{true} y : X \mapsto (Y \mapsto Z_2)
							}
						\end{prooftree}
						\label{B}
					\end{figure}

					Con $R_2 = \{X = \texttt{Bool} \mapsto Z_1, 
					Z_1 = Y \mapsto Z_2\}$.

					Unificando tenemos que $\{X = Bool \mapsto (Y \mapsto Z_2)\}$.

					Por lo que la expresión tiene tipo

					\[
						\lambda x. \lambda y . 
						x \texttt{true} y : (Bool \mapsto (Y \mapsto Z_2)) 
						\mapsto (Y \mapsto Z_2)
					\]


					

        		}
        		\item {
        			$\texttt{snd} = \lambda p.p  \texttt{false} 
        			 = \lambda p.p (\lambda x\lambda y.y) $
        			Como Var($\texttt{and}$) = $\{x$: X, $y$: Y, $p$: P$\}$
        			\begin{align}       		 
        			&\varnothing | \{x: X, y: Y\} \models x : X &\mathrm{(Var)}\\
        			&\varnothing | \{x: X, y: Y\} \models x : X &\mathrm{(Var)}\\
        			&= 0 &\mathrm{(Definicion \ Integral)}
        			\end{align}
				}
			\end{enumerate}
        }
    \end{enumerate}
\end{document}