% tipo
\documentclass{article}

% formato
\usepackage[letterpaper, margin = 1.5cm]{geometry}
\usepackage{float}
\usepackage{nameref}

% matematicas
\usepackage{amsmath}
\usepackage[spanish]{babel}
\usepackage[utf8]{inputenc}
\usepackage{amssymb}
\usepackage{ebproof}
% encabezado

\title{
    Lenguajes de Programación 2020-1\\
    Facultad de Ciencias UNAM\\
    Ejercicio Semanal 8
}

\author{
    Sandra del Mar Soto Corderi\\
    Edgar Quiroz Castañeda
}

\date{
    10 de octubre del 2019
}

\begin{document}
    \maketitle

    \begin{enumerate}
        \item {
		   Encontrar el tipo mas general para las siguientes expresiones de 
		   Cálculo Lambda utilizando el algoritmo de inferencia de tipos W.

		   Como todos los ejercicios incluyen las funciones \texttt{true} y 
		   \texttt{false}, para ahorar algunos pasos se podrían derivar los 
		   tipos de estas con antelación.

		   \begin{figure}[H]
			   \centering
			   \begin{prooftree}
				   \hypo{
					   \Gamma_1 = \{a:A, b:B\} \vdash a:A
				   }

				   \infer 1 [Var] {
					   \varnothing | \Gamma_1 \vdash a:A
				   }
				   \infer 1 [Lam] {
					   \varnothing | \Gamma_2 = \{a:A\}
					   \vdash \lambda b . a : B \mapsto A
				   }

				   \infer 1 [Lam] {
					   \varnothing | \varnothing
					   \vdash \lambda a . \lambda b . a : A \mapsto (B \mapsto A)
				   }
			   \end{prooftree}
			   \caption{Tipo de \texttt{true}}
			   \label{fig:t_true}
		   \end{figure}

		   Y como se llegó a este tipo sin restricciones, entonces ese el tipo
		   de la expresión.

		   \begin{figure}[H]
			\centering
			\begin{prooftree}
				\hypo{
					\Gamma_1 = \{a:A, b:B\} \vdash a:A
				}

				\infer 1 [Var] {
					\varnothing | \Gamma_1 \vdash a:A
				}
				\infer 1 [Lam] {
					\varnothing | \Gamma_2 = \{a:A\}
					\vdash \lambda a . a : A \mapsto A
				}

				\infer 1 [Lam] {
					\varnothing | \varnothing
					\vdash \lambda b . \lambda a . a : B \mapsto (A \mapsto A)
				}
			\end{prooftree}
			\caption{Tipo de \texttt{false}}
			\label{fig:t_false}
		\end{figure}

		Y como se llegó a este tipo sin restricciones, entonces ese el tipo
		de la expresión.s

        	
        	\begin{enumerate}
        		\item {
				 $\texttt{and} = \lambda x\lambda y.xy\texttt{true}$
        		 
				 Primero
				 \begin{figure}[H]
					 \centering
					 \begin{prooftree}

						 \hypo{x:X \in \Gamma_1 = \{x:X\}}
						 \infer 1 [Var] {
							 \varnothing | \Gamma_1 \vdash x:X
						 }

						 \hypo{y:Y \in \Gamma_2 = \{y:y\}}
						 \infer 1 [Var] {
							 \varnothing | \Gamma_2 \vdash y:Y
						 }

						 \hypo{S = \varnothing}

						 \infer [no rule] 1 {Z_1 \texttt{ fresh}}

						 \infer [no rule] 1 {
							 (\varnothing \cup \{X\})
							 \cap
							 (\varnothing \cup \{Y\})
							 = \varnothing
						 }

						 \infer 3 [App] {
							 R_1 = \{X = Y \mapsto Z_1\} 
							 | \Gamma_3 = \Gamma_1 \cup \Gamma_2
							 \vdash x y : Z_1
						 }
					 \end{prooftree}
					 \label{E}
				 \end{figure}

				 Luego
				 \begin{figure}[H]
					 \centering
					 \begin{prooftree}
						 \hypo{Anterior}
						 \infer 1 [App] {
							 R | \Gamma_3 \vdash x y : Z_1
						 }

						 \hypo{
							 \text{\nameref{fig:t_true}}
						 }
						 \infer 1 [Lam] {
							 \varnothing | \varnothing
							 \vdash \texttt{true}:A \mapsto (B \mapsto A)
						 }

						 \hypo{S = \varnothing}

						 \infer [no rule] 1 {Z_2 \texttt{ fresh}}

						 \infer [no rule] 1 {
							 (\varnothing \cup \{X, Y\})
							 \cap
							 (\varnothing \cup \varnothing)
							 = \varnothing
						 }

						 \infer 3 [App] {
							R_2 = \{Z_1 = (A \mapsto (B \mapsto A)) \mapsto Z_2\}\cup R
							| \Gamma_3 \vdash x y \texttt{true} : Z_2
						 }

						 \infer 1 [Lam] {
							 R_2
							 | \Gamma_4 = \Gamma_3 \backslash \{y : Y\} 
							 \vdash \lambda y . x y \texttt{true} : Y \mapsto Z_2
						 }

						 \infer 1 [Lam] {
							 R_2
							 | \varnothing = \Gamma_4 \backslash \{x : X\} 
							 \vdash \lambda x. \lambda y . 
							 x y \texttt{true} : X \mapsto (Y \mapsto Z_2)
						 }
					 \end{prooftree}
					 \label{D}
				 \end{figure}

				 Con $R_2 = \{X = Y \mapsto Z_1, Z_1 = (A \mapsto (B \mapsto A))
				  \mapsto Z_2\}$, que
				 unificado es $\{X = Y \mapsto ((A \mapsto (B \mapsto A)) \mapsto Z_2)\}$

				 Por lo que la expresión tiene tipo

				 \[
					 \lambda x. \lambda y . 
					 x y \texttt{true} : (Y \mapsto ((A \mapsto (B \mapsto A)) 
					 \mapsto Z_2)) \mapsto (Y \mapsto Z_2)
				 \]
        		}
        		\item {
					$\texttt{or} = \lambda x \lambda y.x \texttt{true} y$
					 
					Primero
					\begin{figure}[H]
						\centering
						\begin{prooftree}

							\hypo{x:X \in \Gamma_1 = \{x:X\}}
							\infer 1 [Var] {
								\varnothing | \Gamma_1 \vdash x:X
							}

							\hypo {
								\text{\nameref{fig:t_true}}
							}

							\infer 1 [Lam] {
								\varnothing | \varnothing
								\vdash \texttt{true}:A \mapsto (B \mapsto A)
							}

							\hypo{S = \varnothing}

							\infer [no rule] 1 {Z_1 \texttt{ fresh}}

							\infer [no rule] 1 {
								(\varnothing \cup \{X\})
								\cap
								(\varnothing \cup \varnothing)
								= \varnothing
							}

							\infer 3 [App] {
								R_1 = \{X = (A \mapsto (B \mapsto A)) \mapsto Z_1\} 
								| \Gamma_1
								\vdash x \texttt{true} : Z_1
							}
						\end{prooftree}
						\label{A}
					\end{figure}

					Luego
					\begin{figure}[H]
						\centering
						\begin{prooftree}
							\hypo{Anterior}
							\infer 1 [App] {
								R_1 | \Gamma_1 \vdash x \texttt{true} : Z_1
							}

							\hypo{y:Y \in \Gamma_2 = \{y:y\}}
							\infer 1 [Var] {
								\varnothing | \Gamma_2 \vdash y:Y
							}

							\hypo{S = \varnothing}

							\infer [no rule] 1 {Z_2 \texttt{ fresh}}

							\infer [no rule] 1 {
								(\{X, Z_1\} \cup \{X\})
								\cap
								(\varnothing \cup \varnothing)
								= \varnothing
							}

							\infer 3 [App] {
								R_2 = \{Z_1 = Y \mapsto Z_2\} \cup R_1
								| \Gamma_3 = \Gamma_1 \cup \Gamma_2 \vdash x \texttt{true} y : Z_2
							}

							\infer 1 [App] {
								R_2
								| \Gamma_3 \vdash x \texttt{true} y : Z_2
							}

							\infer 1 [Lam] {
								R_2
								| \Gamma_4 = \Gamma_3 \backslash \{y : Y\} 
								\vdash \lambda y . x \texttt{true} y : Y \mapsto Z_2
							}

							\infer 1 [Lam] {
								R_2
								| \varnothing = \Gamma_4 \backslash \{x : X\} 
								\vdash \lambda x. \lambda y . 
								x \texttt{true} y : X \mapsto (Y \mapsto Z_2)
							}
						\end{prooftree}
						\label{B}
					\end{figure}

					Con $R_2 = \{X = (A \mapsto (B \mapsto A)) \mapsto Z_1, 
					Z_1 = Y \mapsto Z_2\}$.

					Unificando tenemos que $\{X = (A \mapsto (B \mapsto A)) 
					\mapsto (Y \mapsto Z_2)\}$.

					Por lo que la expresión tiene tipo

					\[
						\lambda x. \lambda y . 
						x \texttt{true} y : ((A \mapsto (B \mapsto A)) 
						\mapsto (Y \mapsto Z_2))
						\mapsto (Y \mapsto Z_2)
					\]

        		}
        		\item {
					$\texttt{not} = \lambda p.p  \texttt{false}$
					
					\begin{figure}[H]
						\centering
						\begin{prooftree}
							\hypo{
								p:P \in \Gamma = \{p:P\}
							}
							\infer 1 [Var] {
								\varnothing | \Gamma \vdash p : P
							}

							\hypo {
								\text{\nameref{fig:t_false}}
							}

							\infer 1 [Lam] {
								\varnothing | \varnothing 
								\vdash \texttt{false} : B \mapsto (A \mapsto A)
							}

							\hypo{S = \varnothing}

							\infer [no rule] 1 {Z \texttt{ fresh}}

							\infer [no rule] 1 {
								(\varnothing \cup \{P\})
								\cap
								(\varnothing \cup \varnothing)
								= \varnothing
							}

							\infer 3 [App] {
								R = \{P = (B \mapsto (A \mapsto A)) \mapsto Z\} | \Gamma 
								\vdash p \texttt{false} : Z
							}

							\infer 1 [Lam] {
								R | \varnothing = \Gamma \backslash \{p:P\}
								\vdash \lambda p . p \texttt{false} : P \mapsto Z
							}
						\end{prooftree}
						\label{C}
					\end{figure}

					Y como $R = \{P = (B \mapsto (A \mapsto A)) \mapsto Z\}$ 
					ya está unificado, entonces el tipo de la expresión es

					\[
						\lambda p. p \texttt{false} : ((B \mapsto (A \mapsto A))
						 \mapsto Z) \mapsto Z
					\]

				}
			\end{enumerate}
        }
    \end{enumerate}
\end{document}